\documentclass[twocolumn]{article}
\usepackage[top=0.25in, left=0.25in, right=0.25in, bottom=0.25in]{geometry}

\usepackage{chessboard}
\usepackage{setspace}
\usepackage{seqsplit}
\usepackage{amsmath}
\usepackage{amssymb}
% \usepackage{code}
\usepackage{graphicx}
\usepackage{fancyvrb}
\usepackage{url}
\usepackage{textcomp}

\pagestyle{empty}

\renewcommand\comment[1]{}

\newcommand\st{$^{\mathrm{st}}$}
\newcommand\nd{$^{\mathrm{nd}}$}
\newcommand\rd{$^{\mathrm{rd}}$}
\renewcommand\th{$^{\mathrm{th}}$}
\newcommand\tm{$^{\mbox{\tiny \textsc{tm}}}$}

\newcommand\sfrac[2]{{}\,$^{#1}$\!/{}\!$_{#2}$}

\newcommand\citef[1]{\addtocounter{footnote}{1}\footnotetext{\cite{#1}}\ensuremath{^{\mbox{\footnotesize [\thefootnote]}}}}

\setchessboard{showmover=false}

\begin{document} 

\title{Is this the longest Chess game?}
\author{Dr.~Tom~Murphy~VII~Ph.D.\thanks{
Copyright \copyright\ 2020 the Regents of the Wikiplia
Foundation. Appears in SIGBOVIK 2015 with the {\tt ShortMoveString} of the
Association for Computational Heresy; {\em IEEEEEE!} press,
Verlag-Verlag volume no.~0x40-2A.
\textcurrency 17696
}
}

\renewcommand\>{$>$}
\newcommand\<{$<$}

\date{1 April 2020}

\maketitle

\section{Introduction}

In my experience, most chess games end in a few moves. Players who
seek longevity typically do so by playing many games of chess, rather
than a protracted


Although many people ``know how to play'' chess, almost nobody fully
understands the rules of chess, most authoritatively given by
FIDE~\cite{fiderules}. (See for example Figure~\ref{fig:castle960} for
a minor chess scandal that erupted in 2019 over an obscure corner case
in the rules.) Several of these ``deep cuts'' have to do with game-ending
conditions that were introduced to keep the game from going on forever.

Many chess moves are reversible (e.g. moving the knight forward and
back to its starting position~\cite{survival}), so informal games of
chess could last forever with the players repeating a short cycle. In
AD~1561, Ruy L\'opez added the ``fifty-move rule'' to prevent infinite
games.\footnote{This rule only applied to games started after its
  introduction, so it is possible that some pre-1561 games are still
  in progress and may never end.} This rule (detailed below) ensures
that irreversible moves are regularly played, and so the game always
makes progress towards an end state. Another rule, ``threefold
repetition'' also guarantees termination as a sort of backup plan
(either of these rules would suffice on its own).

% This is good for the insurers and reinsurers of chess tournaments
% (games are guaranteed to end, rather than simply ending with
% probability 1).
So, chess is formally a finite game. This good for computer
scientists, since it means that chess has a trivial O(1) optimal
solution. This allows us to move onto other important questions, like:
What is the longest chess game? In this paper I show how to compute
such a game, and then gratuitously present all of its
% XXX CHECK
17,696 
%
half-moves. Even if you are a chess expert (``chexpert''), I bet you
will be surprised at some of the corner cases in the rules that are
involved.

Speaking of rules, let's first detail the two main rules that limit the
length of the game.

\begin{figure}
  \begin{center}
    \chessboard[setfen=2nr1r1k/pb1n1pqp/1p2p1p1/1N1pN3/5P2/1P4P1/P2PPQ1P/2R1K1RB,showmover=false]
  \end{center}
  \caption{ ({\it Nepomniachtchi -- So, 2019.}) White to move during
    a speed Chess960 (aka ``Fischer Random'') tournament. In this
    variant, the pieces start in different positions, but castling
    rules are such that the king and rook end up on the same squares
    that they would in normal chess. As a result, it is possible for
    the king or rook to not move during castling, or for the
    destination square for the king to already be occupied by the
    rook. Attempting to castle in the position depicted,
    (Figure~\ref{fig:castle960}), grandmaster Ian Nepomniachtchi
    first touched the rook to move it out of the way. However,
    piece-touching rules require that when castling, the player
    first moves the king; but how? The rook is occupying g1! One
    commenter suggested tossing the king into the air, then sliding
    the rook to f1 while the king is airborne, and then watching the
    king land dead center on its target. The arbiter required Nepo
    to make a rook move instead, but this was later appealed, and
    the game replayed. } \label{fig:castle960}
\end{figure}

(Players are collaborating, perhaps playing chess to the death)

\subsection{The seventy-five move rule}

The ``fifty-move rule'' 

Although usually known by the name ``fifty-move rule'' ()

\subsection{Fivefold repetition}

Although only 

% \begin{figure}[htp]
% \input{table-two}
% \caption{Minimal joining strings for every letter (rows) to every other letter (columns).} \label{fig:table}
% \end{figure}

\begin{sloppypar}
\tiny
\setstretch{0.1}
\hyphenpenalty=1
\pretolerance=2
\tolerance=2

\noindent \input{slow}
0--0

\end{sloppypar}

\nocite{chesstego}

\bibliography{chess}{}
\bibliographystyle{plain}

\end{document}
