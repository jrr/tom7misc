\documentclass[twocolumn]{article}
\usepackage[top=0.25in, left=0.25in, right=0.25in, bottom=0.25in]{geometry}

\usepackage{chessboard}
\usepackage{setspace}
\usepackage{seqsplit}
\usepackage{amsmath}
\usepackage{amssymb}
% \usepackage{code}
\usepackage{graphicx}
\usepackage{fancyvrb}
\usepackage{url}
\usepackage{textcomp}

\usepackage{adjustbox}

% lets me explicitly set a. or 1. etc. as enum label
\usepackage{enumitem}

\usepackage[pdftex,
            pdfauthor={Dr. Tom Murphy VII Ph.D.},
            pdftitle={Is this the longest chess game?},
            pdfkeywords={SIGBOVIK, chess}]{hyperref}

\pagestyle{empty}

\renewcommand\comment[1]{}

\newcommand\st{$^{\mathrm{st}}$}
\newcommand\nd{$^{\mathrm{nd}}$}
\newcommand\rd{$^{\mathrm{rd}}$}
\renewcommand\th{$^{\mathrm{th}}$}
\newcommand\tm{$^{\mbox{\tiny \textsc{tm}}}$}

\newcommand\sfrac[2]{{}\,$^{#1}$\!/{}\!$_{#2}$}

\newcommand\citef[1]{\addtocounter{footnote}{1}\footnotetext{\cite{#1}}\ensuremath{^{\mbox{\footnotesize [\thefootnote]}}}}

\setchessboard{showmover=false}

% Define black versions of pieces for inline use. Gross, but it works.
\newcommand{\Pawn}[1][1.3ex]{%
\adjustbox{Trim=4.3pt 2.6pt 4.3pt 0pt,width=#1,margin=0.2ex 0ex 0.2ex 0ex}{\BlackPawnOnWhite}%
}%
\newcommand{\Rook}[1][1.58ex]{%
\adjustbox{Trim=3.2pt 2.2pt 3.2pt 0pt,width=#1,raise=0ex,margin=0.1ex 0ex 0.1ex 0ex}{\BlackRookOnWhite}%
}%
\newcommand{\Knight}[1][1.85ex]{%
\adjustbox{Trim=2.3pt 2.35pt 2.5pt 0pt,width=#1,raise=-0.03ex,margin=0.14ex 0ex 0.14ex 0ex}{\BlackKnightOnWhite}%
}%
\newcommand{\Bishop}[1][1.79ex]{%
\adjustbox{Trim=2.3pt 2pt 2.3pt 0pt,width=#1,raise=-0.12ex,margin=0.1ex 0ex 0.1ex 0ex}{\BlackBishopOnWhite}%
}%
\newcommand{\Queen}[1][2.05ex]{%
\adjustbox{Trim=1.2pt 2.2pt 1.2pt 0pt,width=#1,raise=-0.08ex,margin=0.1ex 0ex 0.1ex 0ex}{\BlackQueenOnWhite}%
}%
\newcommand{\King}[1][1.95ex]{%
\adjustbox{Trim=2pt 2pt 2pt 0pt,width=#1,raise=-0.06ex,margin=0.13ex 0ex 0.13ex 0ex}{\BlackKingOnWhite}%
}%


\begin{document} 

\title{Is this the longest Chess game?}
\author{Dr.~Tom~Murphy~VII~Ph.D.\thanks{
Copyright \copyright\ 2020 the Regents of the Wikiplia
Foundation. Appears in SIGBOVIK 2020 with the {\tt ShortMoveString} of the
Association for Computational Heresy; {\em IEEEEEE!} press,
Verlag-Verlag volume no.~0x40-2A.
\textcurrency 17,697
}
}

\renewcommand\>{$>$}
\newcommand\<{$<$}

\date{1 April 2020}

\maketitle

\section{Introduction}

In my experience, most chess games end in a few moves. If you want to
play a lot of chess moves, you just play a lot of chess games. Still,
there are games that seem to go on foreverrrrrrrrrr. Perhaps the
players are trying to {\em lull each other into a false sense of
security} while {\em waiting for the moment to strike}, or perhaps they
are stalling in a game of {\em Chess to the Death}.

Although many people ``know how to play'' chess, almost nobody fully
understands the rules of chess, most authoritatively given by
FIDE~\cite{fiderules}. (See for example Figure~\ref{fig:castle960} for
a minor chess scandal that erupted in 2019 over an obscure corner case
in the rules.) Several of these ``deep cuts'' have to do with
game-ending conditions that were introduced to avoid interminable
games.

Many chess moves are reversible (e.g.~moving the knight forward and
back to its starting position~\cite{survival}), so informal games of
chess could last forever with the players repeating a short cycle. In
AD~1561, Ruy L\'opez added the ``fifty-move rule'' to prevent infinite
games.\footnote{This rule only applied to games started after its
  introduction, so it is possible that some pre-1561 games are still
  in progress and may never end.} This rule (detailed below) ensures
that irreversible moves are regularly played, and so the game always
makes progress towards an end state. Another rule, ``threefold
repetition'' also guarantees termination as a sort of backup plan
(either of these rules would suffice on its own).

% This is good for the insurers and reinsurers of chess tournaments
% (games are guaranteed to end, rather than simply ending with
% probability 1).
So, chess is formally a finite game. This is good for computer
scientists, since it means that chess has a trivial O(1) optimal
solution. This allows us to move onto other important questions, like:
What is the longest chess game? In this paper I show how to compute
such a game, and then gratuitously present all of its 17,697 
moves. Even if you are a chess expert (``chexpert''), I bet you
will be surprised at some of the corner cases in the rules that are
involved.

Speaking of rules, let's first detail the three main rules that limit
the length of the game. These rules cause the game to end in a draw
(tie) when certain conditions are met.

\begin{figure}[b!]
  \begin{center}
    \chessboard[
      % fillarea=a3-h1,printarea=a3-h1,
      marginbottom=false,margintop=false,
      setfen=2nr1r1k/pb1n1pqp/1p2p1p1/1N1pN3/5P2/1P4P1/P2PPQ1P/2R1K1RB,
      pgfstyle=curvemove,markmoves={e1-g1},
      pgfstyle=straightmove,markmoves={g1-f1},
      showmover=false]
  \end{center}
  \caption{ ({\it Nepomniachtchi -- So, 2019.}) White to move during a
    speed Chess960 (aka ``Fischer Random'') tournament. In this
    variant, the pieces start in different positions, but castling
    rules are such that the king and rook end up on the same squares
    that they would in normal chess. As a result, it is possible for
    the king or rook to not move during castling, or for the
    destination square for the king to already be occupied by the
    rook. Attempting to castle in the position depicted, grandmaster
    Ian Nepomniachtchi first touched the rook to move it out of the
    way. However, piece-touching rules require that when castling, the
    player first moves the king (and ``Each move must be played with
    one hand only''); but how? The rook is occupying g1! One commenter
    suggested tossing the king into the air, then sliding the rook to
    f1 while the king is airborne, and then watching the king land
    dead center on its target. The arbiter required Nepo to make a
    rook move instead, but this was later appealed, and the game
    replayed. } \label{fig:castle960}
\end{figure}

% TODO: (Players are collaborating, perhaps playing chess to the death)

\subsection{The seventy-five move rule} \label{sec:75}

The ``fifty-move rule,''~\cite{wikipedia50move} as it is usually
known, requires that an irreversible move is played at a minimum pace.
For the sake of this rule, irreversible moves are considered captures
and pawn moves,%
\footnote{ Other irreversible moves include: Castling, moving a piece
  so as to lose castling rights, and declining to capture {\it en
    passant} (such capture must be made immediately, so if the option
  is not taken, it cannot be regained). The fifty-move rule could be
  soundly expanded to include these, but, that's just like, not the
  rule, man.}
%
including promotion and {\it en passant} (which is also a capturing
move) but not the movement of a previously promoted piece.
Specifically~\cite{fiderules}:

\begin{quote}
{\bf 9.3}.\quad The game is drawn, upon a correct claim by the player having
the move, if:
  \begin{enumerate}[label=\alph*.]
    \item he writes his move on his scoresheet and declares to the
      arbiter his intention to make this move, which shall result in
      the last 50 moves having been made by each player without the
      movement of any pawn and without any capture, or
    \item the last 50 consecutive moves have been made by each player
      without the movement of any pawn and without any capture.
  \end{enumerate}
\end{quote}

Note that what is provided here is the option for a player to {\em
  claim} a draw (and the two provisions essentially allow either player
to claim the draw at the moment of the 50\th\ move). If neither player
is interested in a draw, either because they think their position is
still winning, or are just trying to create the longest ever chess game,
the game legally continues. That's why what is actually relevant for this
paper is provision 9.6, which defines a draw:

\begin{quote}
  {\bf 9.6}.\quad If one or both of the following occur(s) then the
  game is drawn:

  \ldots 9.6.2. any series of at least 75 moves have been made by each
  player without the movement of any pawn and without any capture. If
  the last move resulted in checkmate, that shall take precedence.
\end{quote}

Draws after 75 moves (per player, so really 150 moves) are compulsory.

Interestingly (at least as interesting as anything in this dubious
affair) it is known that some otherwise winning endgame positions
require more than 50 moves to execute (Figure~\ref{fig:matein545}).
The rules of chess have at various times allowed for longer timers in
such known situations, but were later simplified to the fixed 50 (and
75) move limit.

\subsection{Fivefold repetition} \label{sec:fivefold}

The 75-move rule is rarely applied in practice, but its
counterpart, ``threefold repetition''~\cite{wikipediathreefold} is often
the cause of draws in modern chess. This rule states that if the same
position appears three times, the players can claim a draw:

\begin{quote}
\begin{itemize}
  \item[9.2.2.] Positions are considered the same if and only if the
same player has the move, pieces of the same kind and colour occupy
the same squares and the possible moves of all the pieces of both
players are the same. [...]
\comment{
  Thus positions are not the same if:
\begin{itemize}
\item[9.2.2.1.]	at the start of the sequence a pawn could have been captured en passant
\item[9.2.2.2.]	a king had castling rights with a rook that has not been moved, but forfeited these after moving. The castling rights are lost only after the king or rook is moved.
\end{itemize}
}
\end{itemize}
\end{quote}

Like the 75-move rule, this rule has an optional version (upon
three repetitions) and a mandatory one in 9.6:

\begin{quote}
  [The game is a draw if \ldots] \\
  9.6.1. the same position has appeared, as in 9.2.2, at least five times.
\end{quote}

Fascinatingly (at least as fascinating as anything in this
questionable undertaking), this rule used to require {\em consecutive}
repetition of moves. However, there exist infinite sequences of moves
with no consecutive $n$-fold repetition. For example, in the starting
position, white and black can move either of their knights out and
back. Let {\bf 0} be \knight c3 \Knight c6 \knight b1 \Knight b8
(queenside knights move, returning to the starting position) and {\bf
  1} be \knight f3 \Knight f6 \knight g1 \Knight g8 (kingside). Now
the Prouhet--Thue--Morse sequence~\cite{thue1906uber}
0110100110010110\ldots\footnote{
  Let $s_0$ be the string {\tt "0"}, then
  $s_i$ is $s_{i-1} \overline{s_{i-1}}$
  where $\overline{\tt 0} = {\tt 1}$ and $\overline{\tt 1} = {\tt 0}$.
}
%
can be executed. This infinite
sequence is cube-free (does not contain $SSS$ for any non-empty string
$S$)~\cite{oeisa010060}, and therefore never violates the consecutive
threefold repetition rule~\cite{euwe1929mengentheoretische}.

\medskip
Many implementations of chess ignore these rules or treat them
incorrectly. Implementation of the seventy-five move rule simply
requires a count of how many moves have transpired since a pawn move
or capture, but programs typically do not force a draw after 75 moves.
Repetition requires more work, since the program must keep track of
each of the states reached since the last irreversible move. There are
also some corner cases, such as ambiguity as to whether the starting
position has ``appeared'' before the first move~\cite{survival}. The
ubiquitous FEN notation for describing chess positions does not even
include any information about states previously reached.

\begin{figure}[htp]
  \begin{center}
    \chessboard[setfen=8/r6n/8/8/5k2/3K4/7N/3b3Q b - - 0 1]
  \end{center}
  \caption{Black to move and mate in 545 moves (!). The position was
    found (by Zakharov and Makhnichev~\cite{tablebase7}) while
    building an endgame tablebase of all possible 7-piece positions.
    Of course, the game ends prematurely in a draw because of the
    75-move rule. } \label{fig:matein545}
\end{figure}

\subsection{Dead position} \label{sec:insufficient}

The informal version of this rule (``insufficient material'') states
that if neither side has enough pieces to mate the opponent (for
example, a king and bishop can never mate a bare king) then the game
is drawn. Again, the formal rule is more subtle:

\begin{quote}
5.2.2. The game is drawn when a position has arisen in which neither
player can checkmate the opponent's king with any series of legal
moves. The game is said to end in a `dead position'. This immediately
ends the game\ldots
\end{quote}

This clearly includes the well-known material-based cases like king
and knight vs.~king, but it also surprisingly includes many other
specific positions, especially those with forced captures
(Figure~\ref{fig:insufficient}).

\begin{figure}[htp]
  \begin{center}
    \chessboard[margintop=false,marginbottom=false,setfen=8/8/5K2/8/8/8/Q7/k7 b - - 0 1]
  \end{center}
  \caption{Black to move and draw in 0 (!). Most players and even
    chexperts believe that the only legal move is Kxa2, and
    that the game then ends in a draw with ``insufficient material.''
    However, this game is {\em already over}. Since neither black
    nor white can win via any series of legal moves, by rule 5.2.2
    the game immediately ends in a `dead position'. (Although see
    Section~\ref{sec:ambiguity} for possible ambiguity in this
    rule.)} \label{fig:insufficient}
\end{figure}

The insufficient material rule is curious in that it requires
nontrivial computation to implement. In order to know whether the
position is a draw, an implementation needs to be able to decide
whether or not a series of moves that results in checkmate exists.
Note that this is not nearly as bad as normal game tree search because
the two sides can collaborate to produce the mate (it is not a
$\exists \forall \exists \forall \ldots$ but rather $\exists \exists
\exists \exists \ldots$). Still, such ``helpmates'' can still be quite
deep (dozens of moves) and are interesting enough to be a common
source of chess puzzles. Proving the non-existence of a helpmate can
be very difficult indeed (Figure~\ref{fig:nohelp}).\footnote{Perhaps
  an enterprising reader can prove that for generalized chess, it is
  NP-hard to decide whether the game has ended due to this rule?}

\begin{figure}[htp]
  \begin{center}
    \chessboard[margintop=false,marginbottom=false,setfen=k1bB4/8/2p1p1p1/1pPpPpP1/1P1P1P2/8/8/K1Bb4 w - - 0 1]
  \end{center}
  \caption{Thinking of implementing the rules of chess? To be correct,
    you'll need your program to be able to deduce that no helpmates are
    possible in this position and thus the game is over. Stockfish rates
    this as +0.4 for white, even searching to depth 92.
    (Position is due to user {\tt supercat} on Chess StackExchange.)
  } \label{fig:nohelp}
\end{figure}


\subsubsection{Ambiguity} \label{sec:ambiguity}

Moreover, this rule contains some ambiguity. The phrase ``any series
of legal moves'' is usually taken to mean something like, ``the
players alternate legal moves and follow most of the normal rules of
chess.'' In my opinion it is hard to justify an interpretation like
this.

First, the rules specifically define ``legal move'', with 3.10.1
saying ``A move is {\em legal} when all the relevant requirements of
Articles 3.1 -- 3.9 have been fulfilled.'' These requirements describe
the movement of each piece as you are familiar (e.g. 3.3 ``The rook
may move to any square along the file or the rank on which it
stands.''). They also disallow capturing one's own pieces, or moving
when in check. However, they allow as legal some moves that would
otherwise be prohibited, like capturing the opponent's king (this is
excluded by 1.4.1, outside the definition of ``legal''). Capturing the
opponent's king is generally not useful for demonstrating that
checkmate is possible (Figure~\ref{fig:kingcapture}), so this is
mostly a curiosity.

\begin{figure}
  \begin{center}
    \begin{tabular}{cc}
      \chessboard[smallboard,marginleft=false,marginright=false,setfen=BBKBRRnk/PPPPPPnn/3P1BRR/3PBPPR/3P1P1P/8/8/8 w - - 0 1] &
      \chessboard[smallboard,marginright=false,setfen=BBKBRR1B/PPPPnP1n/3P2RR/3PBPPR/3P1P1P/8/8/8 w - - 0 1] \\
      (a) & (b) \\
  \end{tabular}
  \end{center}

  \caption{In this contrived and impossible position (a), white has many
    easy paths to mate.  All of black's pieces are pinned.
    There is also a ``series of legal moves''
    where black mates white: \bishop xg7++  \bishop xh8?? \Knight xe7++.
    Capturing the king (by moving twice in a row) is a ``legal move''
    (despite not being allowed by other rules), and doing so unpins
    black's knight to deliver a smothered mate (b). Of course, abuse of
    this dubious technical possibility doesn't change the status of
    the position, since we already have a mating sequence by white.
  } \label{fig:kingcapture}
\end{figure}

Second, what is a ``series'' of legal moves? It seems completely
consistent to allow the white player to make several legal moves in
a row, for example. The rules about alternating moves are
again outside the definition of ``legal move'' and ``series'' is
never defined.

We could instead interpret ``any series of legal moves'' as ``taking
the entirety of the rules of chess, any continuation of the game that
ends in checkmate for either player.'' I like this better, although it
creates its own subtle issues. For example, should the position be
considered dead if there is a checkmating sequence, but it requires
entering a fivefold repetition or exhaustion of the 75-move rule? If
so, this would end the game prematurely, and so it has implications
for the longest possible game (Section~\ref{sec:ending}). Even more
esoterically, this interpretation causes the rule to be
self-referential: A sequence must also be allowed by the rule being
defined. A normal person would take the ``least fixed point'' (in the
Kripke sense~\cite{kripke1975outline}) of this self-referential
definition (fewest positions are drawn). But it is also consistent to
interpret it {\em maximally}---in which case the longest chess game is
zero moves!


\medskip
For completeness, note that there are other routes to a draw
(stalemate, draw offers) which we can ignore; it is easy to avoid
these situations when generating the longest game.


\section{Generating the longest game}

It is generally not hard to avoid repeating positions, so the main
obstacle we'll face is the 75-move rule. Let's call an irreversible
move that resets the 75-move counter a {\em critical move}; this is
a pawn move or capturing move (or both). The structure of the game
will be a series of critical moves (I will call these ``critical''
moves) with a maximal sequence of pointless reversible moves in
between them. If we execute the maximum number of critical moves and
make 149 moves (just shy of triggering the compulsory version of
the 75-move rule) between them, then this will be fairly easily
seen as a maximal game.

In fact, for most positions, it is easy to waste 149 moves and return
to the exact same position. So, the strategy for generating the
longest game can mostly be broken into two tasks: Make a game with a
maximum number of critical moves (it can also contain other moves) and
then pad that game out to maximum-length excursions in between its
critical moves.\footnote{The first phase was constructed by hand.
  Software for inserting excursions and checking the result is
  available at:
  \url{sf.net/p/tom7misc/svn/HEAD/tree/trunk/chess/longest.cc} }

\subsection{Maximal critical moves} \label{sec:maxcritical}

Critical moves are pawn moves and captures. There are 16 pawns, each
of which has 6 squares to move into before promoting, so this is
$16\times 6$ critical moves. There are 14 capturable pieces, plus the
16 pawns (they can be captured after promoting); capturing them nets an
additional $16+14$ critical moves for a total of $126$. Each critical
move can be made after a maximum of $75+74$ reversible moves, giving
$(75+74+1) \times 126 = 18900$ moves. The final move would capture the
last piece, yielding a draw due to the remaining kings being insufficient
material (``dead position''). This is our starting upper bound.

We will not quite be able to use the entire critical move budget. If
pawns only move forward in single steps, they will eventually get
blocked by the opposing pawn on the same file. Pawns can move
diagonally off their starting file only by capturing. We have plenty
of capturing to do anyway, so this in no problem. With four captures
per side, the pawns can be doubled, with a clear route to promotion,
like in Figure~\ref{fig:doubled}.

\begin{figure}[h]
\begin{center}
  \chessboard[setfen=2bqkb2/1p1p1p1p/1p1p1p1p/8/8/P1P1P1P1/P1P1P1P1/2BQKB2 w - - 0 1]
  \caption{One way to clear the promotion routes for all pawns with
    eight captures. } \label{fig:doubled}
\end{center}
\end{figure}

However, each of these captures is {\em both} a pawn move {\em and} a capturing
move. This means that we lose $4+4$ critical moves off our
total budget. $150 \times (126 - 8) = 17700$ is the new upper bound.

\paragraph{Parity.} If a critical move (such as a pawn move) is made by
white, then the first non-critical move is made by black. The
players alternate these pointless moves until black has made $75$ and
white $74$. Now it is white's move, and white must make a critical move
or the game ends due to the 75-move rule. If instead we wanted black
to make the next critical move, this would happen after black has made
$74$ and white $74$ non-critical moves. Any time this happens we lose one
move against the upper bound. So, we want to minimize the number of
times we switch which player is making the critical moves. Obviously
we must switch at least once, because both black and white must
make critical moves. This reduces the upper bound to $17699$.

\paragraph{Starting condition.} The first critical move should be made
by black. The starting position (with white to move) is analogous to
the situation just described, as if black has just made a critical move,
and it's white's turn. White will play $75$ reversible moves, black plays
$74$, and the 150\th\ move is black making the first critical move.

Note that white is quite constrained during this beginning phase, as
pawn moves are critical moves and must be avoided. Only the white
knights can escape the back rank. When we try to insert $149$
pointless moves, we'll only be able to move the knights and rooks, and
doing this $75$ times must leave e.g.~one of the knights on an
opposite-colored square.\footnote{Knight moves always change the color
  of the knight's square, and same for rook moves constrained to the
  a1/b1 or g1/h1 squares.} So we have to be a little careful about the
position in which we make black's first critical move.

Since white can only free their two knights, these are the only pieces
that can be captured by black pawns. So it will not be possible for
black to double four sets of their pawns as in
Figure~\ref{fig:doubled}. This would require white to have a phase of
critical moves to free pieces to capture, then black again to finish
doubling pawns, then white again to promote its freed pawns. Each switch
costs one move off the na\"ive max. We can be more efficient with an
asymmetric approach. 

Black's first phase of critical moves instead results in this:

\comment{
1. Nf3 b6 2. Nc3 g6 3. Ne5 g5 4. Nd5 b5
5. Nc4 bxc4 6. Nf4 gxf4 7. Rg1 c6 8. Rh1 c5
9. Rg1 f6 10. Rh1 f5 
}
\begin{center}
  \chessboard[
    margintop=false,marginbottom=false,
    setfen=rnbqkbnr/p2pp2p/8/2p2p2/2p2p2/8/PPPPPPPP/R1BQKB1R w Qkq - 0 11,
    pgfstyle=straightmove,markmoves={a6-b7,c3-b4,d6-c7,e6-f7,f3-g4,h6-g7},
  ]
\end{center}

\smallskip
\noindent The white b and g pawns have a clear route to promotion. White can
free each of the remaining 6 pawns with a single capture. Between this
and black's two pawn captures, this is the optimal 8 pawn moves
that are also captures. Since black has plenty of freed pieces, white
can promote all of their pawns during their own phase of critical
moves, resulting in:

\comment{
11. b3 Nc6 12. b4 Nd4
13. b5 Nb3 14. cxb3 Bb7 15. a3 Rc8 16. a4 Rc6
17. a5 Rd6 18. a6 Re6 19. axb7 Rd6 20. b8=N Re6
21. b6 Nh6 22. b7 Ng4 23. Nc6 Bh6 24. b8=Q Kf8
25. b4 Ke8 26. b5 Kf8 27. Qe5 Ke8 28. b6 Kf8
29. b7 Ke8 30. b8=N Bg5 31. h3 Bh4 32. hxg4 Bg3
33. fxg3 Qc8 34. g5 Qa6 35. Nb4 Kd8 36. Qe3 Kc7
37. g6 Kb6 38. g7 Rf6 39. g8=Q Re6 40. g4 Rf6
41. g5 Re6 42. g6 Rf6 43. g7 Re6 44. Qf8 Rf6
45. g8=N Rd6 46. Nf6 Qb5 47. Nh5 Rg8 48. g3 Rgg6
49. g4 Rge6 50. g5 Rf6 51. g6 Rc6 52. g7 Qa6
53. g8=N Qa5 54. d3 Qa6 55. d4 Qa5 56. d5 Qa6
57. dxc6 Qa5 58. c7 Qa6 59. c8=R Qa4 60. Qh3 Qa6
61. e3 Qa5 62. e4 Qa6 63. e5 Qa5 64. exf6 Qa6
65. Qg7 Qa5 66. f7 Qa6 67. Re8 Qa5 68. f8=R Qa6
69. Qdg4 Qa5 70. Nc6 Qa6 71. Ne5 Qa5 72. Nf3 Qa6
73. Nd3 Qa5+ 74. Ke2 Qa6 75. Nf2 Qa5 76. Qh8 Qa3
77. Nd1 Qa2+ 78. Nd2 Qa3 79. Kf2 Qa2 80. Kg1 Qa3
81. Rxa3
}
\begin{center}
  \chessboard[
    margintop=false,marginbottom=false,
    setfen=4RRNQ/p2pp2p/1k6/2p2p1N/2p2pQ1/R6Q/3N4/2BN1BKR b - - 0 81]
\end{center}

\smallskip
\noindent Now black can promote all of its pawns and capture white's pieces.
We actually leave the white queen; this turns out to be essential:

\comment{
\ldots f3 82. Kh2 f2 83. Bd3 f1=R 84. Qf4 Rxf4
85. Ng7 h6 86. Be4 Rxe4 87. Ne6 f4 88. Nf6 f3
89. Ng8 f2 90. Nf6 f1=B 91. Rd3 Bxd3 92. Bb2 a6
93. Bc3 a5 94. Bb2 a4 95. Bc3 a3 96. Bb2 a2
97. Bc3 a1=R 98. Ba5+ Rxa5 99. Nf4 c3 100. Ne2 c2
101. Nf4 c1=N 102. Ne2 c4 103. Ng4 Nxe2 104. Nge3 c3
105. Nf5 c2 106. Nfe3 c1=R 107. Re1 h5 108. Qf3 h4
109. Kg2 h3+ 110. Kf2 h2 111. Qfh5 h1=R 112. Ng4 Rxg4
113. Nc4+ Bxc4 114. Nb2 d6 115. Rg1 d5 116. Rg2 d4
117. Kf3 d3 118. Kf2 d2 119. Qh2 d1=R 120. Kf3 Nc3
121. Kf2 Rg8 122. Kf3 e6 123. Kf4 e5+ 124. Kf5 e4+
125. Kf6 e3 126. Rg3 e2 127. Rg2 e1=Q 128. Nd3 Rxd3
129. Rg1 Rgxg1 130. Q8h3 Rxh3 131. Ra8 Rxa8 132. Rb8+ Rxb8
}

\begin{center}
  \chessboard[
    margintop=false,marginbottom=false,
    setfen=1r6/8/1k3K2/8/2b5/2n4r/7Q/2r1q1rr w - - 0 133]
\end{center}

\smallskip
\noindent Finally, white captures all of black's pieces, and mates the black king:

\comment{
133. Qxb8+ Ka5 134. Qg3 Bf7 135. Qxe1 Rg5 136. Qxc1 Ka6
137. Qxc3 Kb6 138. Qxh3 Kc5 139. Kxf7 Kc6 140. Qxh1+ Kd7
141. Kf6 Ke8 142. Qg2 Rg8 143. Ke6 Rg7 144. Qxg7 Kd8
145. Qd7\#
}

\begin{center}
  \chessboard[
    margintop=false,marginbottom=false,
    setfen=3k4/3Q4/4K3/8/8/8/8/8 b - - 2 145]
\end{center}

\smallskip
\noindent Since we switch which color is making critical moves a total of three
times, we must come shy of the na\"ive maximum of 17,700 moves by
three. This gives us an upper bound of 17,697 for this approach,
which we will be able to achieve.

\paragraph{Ending condition.} \label{sec:ending} The way this game ends
is subtle for several reasons. First, note that we left the white
queen on the board and used it for mate. It is required that white be
the one mating for parity reasons, similar to the reason black must
make the first critical move. Since white makes critical moves in the
last phase, black leads on non-critical moves; at the moment white
makes the checkmating move, black has made 75 non-critical moves, and
white 74. White's 75\th\ move mates.

But doesn't this trigger the 75-move rule? No, this rule (9.6.2;
Section~\ref{sec:75}) has a special exception for checkmate: ``If the
last move resulted in checkmate, that shall take precedence.''
Essentially, we can treat checkmate as a type of critical move.

Why not capture the white queen too? This would be a critical move,
but once two kings remain, the game ends immediately in a draw. So
there is no length advantage here over checkmate. In fact, attempting
to capture may {\em shorten} the game: Consider the position in
Figure~\ref{fig:insufficient} where black is {\em forced} to capture
the queen. This game is over prior to the capture, so this is shorter
than the checkmating sequence! Even if the king had an escape square,
it is arguable (Section~\ref{sec:ambiguity}) that the game is over in
any case: There is no ``series of legal moves'' that leads to mate.
Either the king captures the queen (and then clearly no mate is
possible with just the two kings) or the king escapes, but in doing so
plays the 75\th\ non-critical move, and triggers a draw by the 75-move
rule.

The most foolproof way to ensure that mate is always possible is for the
game itself to end in mate. This sidesteps any ambiguity about the way
the 75-move condition should be interpreted, as well.


\subsection{Inserting excursions} \label{sec:excursions}

The game described in Section~\ref{sec:maxcritical} has 118 critical
moves in 289 total moves. There is some inefficiency between the
critical moves, but this doesn't matter since we are trying to generate
a long game anyway. In fact, the next step will be to add {\em as much
  inefficiency as possible} in between the critical moves.

% Diagram here or earlier would probably help!

Each ``critical section,'' which is the series of moves ending in a
critical move, can be treated independently. If black ends the section
with a critical move, then we want $75+74$ non-critical moves to be
played, and then black's critical move. For white, of course, $74+75$.
There are many ways we could try to make these $149$ moves; we don't
even have to use the moves that are already there as long as we end
up in the right position to make the critical move. But a simple
approach suffices.

For each critical section, we loop over all of the positions
encountered, and attempt inserting excursions that return us to the
same position but waste moves. There are two types of excursions we
try: Even excursions (each player makes two moves) and odd (each
player makes three moves).

\paragraph{Even excursions.} This four move sequence moves two pieces
of $X_1$ and $X_2$ of opposite colors. $X_1$ moves from $s_1$ to $d_1$
then $X_2$ from $s_2$ to $d_2$, then $X_1$ moves from $d_1$ back to
$s_1$, and $X_2$ from $d_2$ back to $s_2$. Easy. Any piece can perform
this maneuver other than pawns (which would be critical moves anyway)
as long as there are legal squares (considering check, etc.). All of
$s_1$, $d_1$, $s_2$, $d_2$ must be distinct. No shorter excursions are
possible.

\paragraph{Odd excursions.} This is the straightforward extension to
three squares ($s_i \rightarrow m_i \rightarrow d_i \rightarrow s_i$),
for a six-move sequence. The squares for each piece must be distinct
but it is possible for e.g.~$m_2$ to equal $s_1$. Knights cannot
perform this trick; each move changes the color of the square the
knight sits on, which causes a contradiction with a cycle of length
three. All other pieces can do it with sufficient room. The king, for
example, can move horizontally, then diagonally, then vertically back
to its starting square.

Note that odd excursions are not possible early in the game (prior to
white moving any pawns), because even when the knights are free, the
rooks only have two squares (and thus the same color parity argument
applies as knights). Some opportunity can be created by having a black
knight capture one of white's bishops. Fortunately, we do not need any
odd excursions at this point in the game.

We find excursions by just looping over possible moves that satisfy
the criteria, prioritizing odd excursions if the target (divided by
two) is odd. In order to avoid triggering the fivefold repetition
rule, we also keep track of all of the positions encountered, and
never enter a position more than two times. (Here we avoid even
threefold repetition.) It is not necessary to look beyond the critical
section, because critical moves make it impossible to return to a
prior position.

This process is not at all guaranteed to work; it may fail to fill the
critical section. Indeed, as discussed in
Section~\ref{sec:maxcritical}, we must have at least one move of slack
whenever we switch from a critical move by one player to the other.
In practice this approach succeeds readily, and manages to waste 149
moves in each critical section of the input game, save for the three
times that parity requires one move of slack. The full game is uselessly
included in a very tiny font in Section~\ref{sec:longest}.

\section{Reader's guide}

The paper demonstrates a game with three ``switches'' of which side is
making critical moves; each costs a move against the na\"ive maximum
due to parity. We clearly need at least one switch (both sides must
make critical moves), but is it possible to do it with only two? If
not, can this lower bound be proved?

\smallskip
The game given is believed to be maximal, as measured in the number of
moves. But, other metrics exist. For example, the letter g is slightly
wider than f, so moving \Queen g3 is {\em typographically} longer than
\Bishop f1. PGN format itself can be stretched by making moves that
need to be disambiguated (\bishop ff7 means ``move the \bishop\ on the
f file to f7'' (not ``the giant sword from Final Fantasy 7'', as many
believe)) or checking the opponent's king for a bonus +. Some moves
are longer in terms of distance traveled; moving the queen or bishop
between opposite corners is $\sqrt{2} \times 7$ squares! What is the
longest game according to these or other metrics?

\smallskip
In {\it Chess}, it is impossible to capture your own pieces. How does
this limitation apply to your own life?

\smallskip
In many games of {\it Chess}, the black and white pieces are found to
disagree. What does this say about society?

\smallskip
How do you feel about the ending of the game? Is it disappointing that
the result is not symmetric (i.e.~a draw), or elegant that it
demonstrates its own avoidance of the ``dead position'' rule? Was it
what you expected?

\smallskip
The character called \queen\ often plays an important role in the story.
How would you describe her personality? How does she develop over the
course of the game?

\smallskip
{\it Is this the longest Chess game?} is the author's sixth paper
about chess for SIGBOVIK, but it is widely believed that nobody wants
to read this kind of thing. What is wrong with him?

\section{A longest game} \label{sec:longest}

There are jillions of possible games that satisfy the description
above and reach 17,697 moves; here is one of them.\footnote{
  It can also be downloaded at \url{tom7.org/chess/longest.pgn}.
  Many chess programs fail to load the whole game, but this is
  because they decided not to implement the full glory of chess.}
%
Critical moves are marked with {\large \bf [bold]}. Note that in the
standard PGN format for listing games, move numbers are ``full moves''
consisting of a move by white and then black, whereas we use ``move''
in this paper to mean ``half-move''; an individual move by one of the
players. Thus the game ends during full move 8,849, one half of
17,697.

\medskip

\begin{sloppypar}
\tiny
\setstretch{0.1}
\hyphenpenalty=1
\pretolerance=2
\tolerance=2

\noindent \input{slow}
\quad 1-0

\end{sloppypar}

\nocite{chesstego}

\bibliography{chess}{}
\bibliographystyle{plain}

\end{document}
