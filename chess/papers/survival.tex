\documentclass[twocolumn]{article}
\usepackage[top=1.1in, left=0.85in, right=0.85in]{geometry}

\usepackage{url}
% \usepackage{code}
% \usepackage{cite}
\usepackage{amsmath}
\usepackage{amssymb}
\usepackage{graphicx}
\usepackage{chessboard}

\pagestyle{empty}

\usepackage{ulem}
% go back to italics for emphasis, though
\normalem

\usepackage{natbib}

% \newcommand\comment[1]{}
\newcommand\sfrac[2]{\!{}\,^{#1}\!/{}\!_{#2}}

\begin{document} 

\title{Survival in chessland}
\author{Dr.~Tom~Murphy~VII~Ph.D.\thanks{
Copyright \copyright\ 2018 the Regents of the Wikiplia
Foundation. Appears in SIGBOVIK 2018 with the threefold
repetition of the Association for Computational Heresy; 
{\em IEEEEEE!} press, Verlag-Verlag volume no.~0x40-2A.
53 Centipawns} }

\setchessboard{showmover=false}

\newcommand\checkmate{\hspace{-.05em}\raisebox{.4ex}{\tiny\bf ++}}

\renewcommand\th{\ensuremath{{}^{\textrm{th}}}}
\newcommand\st{\ensuremath{{}^{\textrm{st}}}}
\newcommand\rd{\ensuremath{{}^{\textrm{rd}}}}
\newcommand\nd{\ensuremath{{}^{\textrm{nd}}}}
\newcommand\at{\ensuremath{\scriptstyle @}}

\date{1 April 2018}

\maketitle \thispagestyle{empty}

\begin{abstract}
CHESSMATE.
\end{abstract}

\section*{Introduction}


If you are forced to play chess to the death, you are in trouble,
because most people are not good at chess (for example, the author)
and yet want to live.\footnote{It is easy for two players to
  collaborate to produce a draw, especially by simply agreeing to a
  draw at the outset of the game (if allowed). Some tournament formats
  forbid the players from agreeing to a draw verbally before a certain
  point (e.g. 30 moves). There are always other routes to a draw, for
  example by stalemate or repeating the same position three times.
  Collaboratively producing such situations is easy, but this strategy
  is not likely a stable equilibrium: Players can often gain a
  substantial advantage by going ``off script'' and instead trying to
  win the game. Additionally, sometimes the terms of
  chess-to-the-death do not allow the players to communicate at all
  beforehand, nor during the game. If this is the case, then it may be
  difficult to agree on the approach to drawing, let alone establish
  that this is both player's desire. Since the rules of
  chess-to-the-death can't forbid us from colluding at this moment, I
  hereby declare that the following is the correct approach: 
%
  1. Nf3. This is a reasonable opening move for white (``Zukertort opening'')
  which can transpose into several common systems (e.g. King's
  Indian). Since the knight can move back to g1 on the next move,
  knight moves are the fastest route to a draw by repetition. This
  has a good chance of signaling to a wise player that a draw is
  desired. The player should make this move after pondering carefully
  for some time, and then looking meaningfully into the other player's
  eyes. 
  1. ... Nf6. This is both a strong response for black in a real game,
  and simultaneously a signal that a draw is desired. The other
  advantage is that very weak players may simply copy what white does.
  In doing so, they will also play this move. 
  2. Ng1?!. This is a terrible move for white, but clearly signals
  the intention to draw.
  2. ... Ng8!. ``Fool's Draw Accepted.'' The starting position is reached
  for a second time.
  3. Nf3 Nf6. At this point the game should clearly continue repeating
  the sequence, although since we are in the start position, white has
  any number of strong opening moves available. Signaling the draw line
  and then 3. d4!? may be pyschologically devastating.
  4. Ng1 Ng8 1/2-1/2. The starting position is reached for the third
  time, which by rule is a draw.

  Since this is one of the shortest possible routes to a draw, I hereby
  dub this line the ``Fool's Draw,'' by analogy with the Fool's Mate.

  If 1. ... Nc6 or another Knight's move, white can also consider
  continuing in the obvious way. However after 1. ... d5, black
  has refused or not noticed the draw. Fortunately, white is still
  in a good position to play the game normally. White can try
  to be more obvious with 2. Ng1, but if black is choosing to just
  play normally, white takes a distinct handicap by doing so.

  The biggest risk for white is that black does not play 2. ... Ng8
  but rather a normal move like 2. ... g6 (``Fool's Draw Betrayed'').
  This can happen if black is not metagaming at all (for Nf6 is a
  normal response to the normal Nf3), or if black is an exceptionally
  shrewd metagamer (tricking white into wasting two tempos with Ng1
  by pretending to be cooperating).

  Of course, this all relies on the assumption that if
  chess-to-the-death ends in a draw, the players are spared or allowed
  to repeat indefinitely. If both players are actually executed, then
  this line is truly a Fool's Draw! }
%
But what if you are forced to {\it be one of the chess pieces} to the
death? That is, your little soul inhabits one of the 32 pieces or
pawns and your soul is vanquished if that piece is eliminated. Now it
doesn't matter whether you're good or bad at chess, because you don't
get to pick what happens in the game. What matters is that your piece
lives to the end of the game, when all surviving pieces are set free.
Which piece should you want to be?

In formal chess, the king can never be captured: The game ends when
the king is attacked but cannot move, and it is illegal to make a move
that leaves the king attacked. The king's death is implied, of course,
but it is seen as more poetic to end the game prior to this point.

For the sake of this question, we'll consider the the white king to
``die'' if white loses (i.e., is checkmated), and likewise for black.
Otherwise, of course, the best chances of survival would trivially be
with the two kings, since they always survive. Loss includes
resignation, since most high-level games actually end once the
defeated player agrees that loss is inevitable. We can think of this
common case like king seppuku. Many games also end in time forfeit,
which is like the king's poor diet and lifestyle choices leading to
a death by natural causes.

Neither side has a decisive advantage, and many games end in a draw,
with both kings surviving. So the survival chances of a king are
clearly greater than 50\%---pretty decent odds. Is it possible that
any other piece has even better chances?

\section{Hypotheses}

Like all good scientific research, I clearly laid out my hypothesis
and wrote down the motivation before performing the study. This helps
prevent presentation bias where the results appear more satisfactory
because they are framed as a natural conclusion from the idea that
motivated the research in the first place (when in fact, of course,
if you write the motivation after witnessing the results, ..). 
It is also much more exciting. I literally don't know the answer
as I'm writing this, nor whether it is interesting in any way!

Here are my guesses.

\begin{itemize}
\item Black and white are probably not significantly different. That
  is, the a1 and a8 rooks probably have about the same survival
  chances (it's known that white has a slight statistical advantage
  but it is probably only around 1\%). So these guesses will be
  written about white's pieces.
\item The d2 and e2 pawns are very active in common openings, and
  are frequently captured as part of those openings. I think they
  are the least likely to survive overall.
\item Bishops and knights are often involved in the opening and
  midgame, and often exchanged nonchalantly. I think they all have
  relatively low survival chances.
\item Although the queen is very valuable, a queen exchange is often
  forced for games that enter the endgame.
\item Rooks tend to be late-game pieces, because they are difficult
  to get out of their corners (and at most one can be activated
  by the fastest method, castling) and are relatively valuable.
\item This leaves the non-central pawns. These are the hardest to
  predict, and they are hard to think about (at least for me) because
  when e.g. the a2 pawn recaptures the b3 pawn that it supported, I
  just think of this as the b3 pawn. Of these pawns, b2 and g2 are
  somewhat weak because they are undefended once the bishop is
  developed (cf. the famous ``poison pawn'' at b2). On the other hand,
  in the fianchetto configuration, this pawn is very strong and often
  survives the entire game without leaving the third rank. Since pawn
  chains usually progress towards the middle of the board, the a2 pawn
  is more likely to be supporting than supported. This both leaves it
  weak to capture, but prone to recapturing. Outside pawns block one's
  own rook, although for this same reason they often clear the file by
  capturing (and so survive). They are also commonly used to push into
  a well-defended king's territory (e.g. in the fianchetto); kingside
  castling is more common, so this means that the h pawns are often
  lost to this fate.
\end{itemize}

This leaves my final ranking, from most surviving to most dead:

\begin{enumerate}
\item  f2
\item  c2
\item  g2
\item  a2
\item  h2
\item  b2,
\item  h1 rook,
\item  a1 rook,
\item  king,
\item  queen,
\item  f1 bishop,
\item  c1 bishop,
\item  g1 knight,
\item  b1 knight,
\item  e2,
\item  d2,
\end{enumerate}

As already copped to, while the author is an aficionado and also knows
how to spell the difficult word aficionado without spell-check, he is
not good at chess. A few drinking buddies with varying chessperience were
also consulted for their wagers.

\subsection{Ben}

edge pawns almost never played til endgame let alone traded off

ph
pa

not quite sure where these should go (pb more likely to see play in
queenside minority attacks in k-side castle games?)

pf
pg
pb

rook play more likely to be active on q side than on k side (also the
classic Nxc7 fork in low rank play), but overall more likely to stay
tucked away compared to q

Rh
Ra

i think IQP positions are more likely than not saccing e in e4 openings
but on the other hand d is often traded off in e4 openings while vice
versa is not as true

pe
pd

q probably involved in many checkmates (low ranked play) or resignations
before traded off (high ranked play)

Q

just randomly guessing k dies in about 1/3 of games, times 1/2 for 2
sides

K

this pawn is a super goner (sicilian, QGA, ...)

pc

most doomed seem to be the minor pieces as i'd guess at least half of
them get traded off on c/f/3/6 or e/d/4/5 in near every game so

Bc
Bf
Nb
Ng

\subsection{Jim}

Most-to-least-survival hero tier list for chess (patch 1.0):

1.: King
 -- If I estimate that about 2/3rds of all regular pieces are captured in an average game, and the probability of any non-king piece being captured is uniform, then the king is clearly the most likely to survive. (I'm going to break symmetry here and rank black king less likely to survive than white king.)

2. Both Rooks
 -- Kept in reserve for castling purposes.

3. The A, B, G, and H pawns
 -- maybe people will forget to move them because they are far from the center.

4. Both Knights
 -- They are slippery, but they often get deep into enemy territory quickly.

5. The C,D,E, F pawns
 -- Moved forward to release various more important pieces => more likely to die.

6. Both Bishops
 -- https://www.youtube.com/watch?v=gDnE-5lD7w8

7. The Queen
 -- A high-value target, seems unlikely to survive.

This list was assembled for chess.report by ix, who only barely understands the rules of chess and rarely plays. This is probably why the justifications get increasingly nonsensical.

\subsection{David}

Here's my ranking, along with the probabilities of survival that I guess for each piece:

white g-pawn: 0.72
black g-pawn: 0.72
white b-pawn: 0.69
black a-pawn: 0.66
black b-pawn: 0.65
white king: 0.65
white a-pawn: 0.64
black king: 0.55
black f-pawn: 0.54
white f-pawn: 0.54
black h-pawn: 0.53
white h-pawn: 0.53
white kingside rook: 0.52
black kingside rook: 0.52
white queenside rook: 0.51
black queenside rook: 0.51
white c-pawn: 0.49
black c-pawn: 0.44
white light-squared bishop: 0.33
black light-squared bishop: 0.33
white dark-squared bishop: 0.32
black dark-squared bishop: 0.32
white queen: 0.31
black queen: 0.30
white kingside knight: 0.29
black kingside knight: 0.29
white queenside knight: 0.28
black queenside knight: 0.28
white e-pawn: 0.19
black d-pawn: 0.18
white d-pawn: 0.17
black e-pawn: 0.16

\subsection{William}

Okay, this is pretty much off the cuff, but let’s say, just for white, from most to least likely to survive:

Queenside Rook
B Pawn
A Pawn
Kingside Bishop
C Pawn
Queen
King
H Pawn
Queenside Bishop
F Pawn
G Pawn
Queenside Knight
Kingside Rook
Kingside Knight
D Pawn
E Pawn

Some motivated reasoning:

I figure the king has got to be somewhere near the middle of the pack since he dies in half of games featuring a winner — but with slightly higher-than-even odds of surviving, since some games end in a draw.  I’m probably mixing up means and medians here somehow..

...
King
...

I’m gonna assume castling happens more often on the King’s side, so let’s give Kingside Rook and F, G, and H Pawns a better shot than their fellows on the left.  But maybe it should actually be worse, since if they die, it’s because they failed to protect the king.  Plus, having heard the tip about C Pawn loud and clear, I’m gonna assume that bad boy most often becomes a new Queen, which means he gets more survival points than the real Queen herself. 

...
C Pawn
Queen
A Pawn
B Pawn
Queenside Rook
King
F Pawn
G Pawn
H Pawn
Kingside Rook
...

D and E Pawn are nothing but pawns, and they mostly sacrifice themselves to the cause.

...
D Pawn
E Pawn

Randomizing within these constraints gives us our starting point.  Then the wildcard Bishops and Knights get randomly distributed through what remains to come up with this final answer shown above.


% Battle chess
% Feminist game 

\nocite{chesstego}

\bibliography{chess}{}
\bibliographystyle{plain}

\end{document}
