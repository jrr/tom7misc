\documentclass[twocolumn]{article}
\usepackage[top=0.5in, left=0.45in, right=0.45in, bottom=0.5in]{geometry}

\usepackage{url}
% \usepackage{code}
% \usepackage{cite}
\usepackage{amsmath}
\usepackage{amssymb}
\usepackage{graphicx}
\usepackage{chessboard}

% \usepackage{chessfs}
\usepackage{adjustbox}

% Define black versions of pieces for inline use. Gross, but it works.
\newcommand{\Pawn}[1][1.3ex]{%
\adjustbox{Trim=4.3pt 2.6pt 4.3pt 0pt,width=#1,margin=0.2ex 0ex 0.2ex 0ex}{\BlackPawnOnWhite}%
}%
\newcommand{\Rook}[1][1.58ex]{%
\adjustbox{Trim=3.2pt 2.2pt 3.2pt 0pt,width=#1,raise=0ex,margin=0.1ex 0ex 0.1ex 0ex}{\BlackRookOnWhite}%
}%
\newcommand{\Knight}[1][1.85ex]{%
\adjustbox{Trim=2.3pt 2.35pt 2.5pt 0pt,width=#1,raise=-0.03ex,margin=0.14ex 0ex 0.14ex 0ex}{\BlackKnightOnWhite}%
}%
\newcommand{\Bishop}[1][1.79ex]{%
\adjustbox{Trim=2.3pt 2pt 2.3pt 0pt,width=#1,raise=-0.12ex,margin=0.1ex 0ex 0.1ex 0ex}{\BlackBishopOnWhite}%
}%
\newcommand{\Queen}[1][2.05ex]{%
\adjustbox{Trim=1.2pt 2.2pt 1.2pt 0pt,width=#1,raise=-0.08ex,margin=0.1ex 0ex 0.1ex 0ex}{\BlackQueenOnWhite}%
}%
\newcommand{\King}[1][1.95ex]{%
\adjustbox{Trim=2pt 2pt 2pt 0pt,width=#1,raise=-0.06ex,margin=0.13ex 0ex 0.13ex 0ex}{\BlackKingOnWhite}%
}%

\interfootnotelinepenalty=0

% lets me explicitly set a. or 1. etc. as enum label
\usepackage{enumitem}

\pagestyle{empty}

\usepackage{ulem}
% go back to italics for emphasis, though
\normalem

\usepackage{natbib}

\setlength{\footnotesep}{2em}

% \newcommand\comment[1]{}
\newcommand\sfrac[2]{\!{}\,^{#1}\!/{}\!_{#2}}

\begin{document} 

\title{Survival in chessland}
\author{Dr.~Tom~Murphy~VII~Ph.D.\footnote{
Copyright \copyright\ 2019 the Regents of the Wikiplia
Foundation. Appears in SIGBOVIK 2019 with the threefold
repetition of the Association for Computational Heresy; 
{\em IEEEEEE!} press, Verlag-Verlag volume no.~0x40-2A.
53 Centipawns} }

\setchessboard{showmover=false}

\newcommand\checkmate{\hspace{-.05em}\raisebox{.4ex}{\tiny\bf ++}}

\renewcommand\th{\ensuremath{{}^{\textrm{th}}}}
\newcommand\st{\ensuremath{{}^{\textrm{st}}}}
\newcommand\rd{\ensuremath{{}^{\textrm{rd}}}}
\newcommand\nd{\ensuremath{{}^{\textrm{nd}}}}
\newcommand\at{\ensuremath{\scriptstyle @}}

\date{1 April 2019}

\maketitle \thispagestyle{empty}

\begin{abstract}
CHESSMATE.
\end{abstract}

\section*{Introduction}


If you are forced to play chess to the death, you are in trouble,
because most people are not good at chess (for example, the author)
and yet want to live.\footnote{It is easy for two players to
  collaborate to produce a draw, especially by simply agreeing to a
  draw at the outset of the game (if allowed). Some tournament formats
  forbid the players from agreeing to a draw verbally before a certain
  point (e.g. 30 moves), or without the arbiter's consent, and FIDE
  rules technically do not allow a draw until both players have made a
  move (5.2.3). There are always other routes to a draw, for example
  by stalemate or repeating the same position three times.
  Collaboratively producing such situations is easy, but this strategy
  is not likely a stable equilibrium: Players can often gain a
  substantial advantage by going ``off script'' and instead trying to
  win the game. Additionally, sometimes the terms of
  chess-to-the-death do not allow the players to communicate at all
  beforehand, nor during the game. If this is the case, then it may be
  difficult to agree on the approach to drawing, let alone establish
  that this is both players' desire. Since the rules of
  chess-to-the-death can't forbid us from colluding right now as you
  read this paper, I hereby declare that the following is the correct
  approach:
\begin{enumerate}[label=\arabic*.]
  \item[1.] Nf3. This is a reasonable opening move for white (begins the R\'eti)
    which can transpose into several common systems (e.g. King's
    Indian). Since the knight can move back to g1 on the next move,
    knight moves are the fastest route to a draw by repetition. This
    has a good chance of signaling to a wise player that a draw is
    desired. The player should make this move after pondering carefully
    for some time, and then looking meaningfully into the other player's
    eyes. 
  \item[1.] \ldots Nf6. This is both a strong response for black in a real game,
    and simultaneously a signal that a draw is desired. The other
    advantage is that very weak players may simply copy what white does.
    % XXX potential citation to 'elo world' if players are stateful..
    In doing so, they will also play this move. 
  \item[2.] Ng1?!. White moves the knight back to its starting square.
    This is a terrible move for white, but clearly signals the intention to
    draw.
  \item[2.] \ldots Ng8!. ``Fool's Draw Accepted.'' The starting position is reached
    for a second time.
  \item[3.] Nf3 Nf6. At this point the game should clearly continue repeating
    the sequence, although since we are in the start position, white has
    any number of strong opening moves available. Signaling the draw line
    and then 3. d4!? may be pyschologically devastating.
  \item[4.] Ng1 Ng8 1/2-1/2. The starting position is reached for the third
    time, which by rule\footnotemark\ is a draw.
\end{enumerate}

  Since this is one of the shortest possible routes to a draw, I hereby
  dub this line the ``Fool's Draw,'' by analogy with the Fool's Mate.

  If 1. \ldots Nc6 or another Knight's move, white can also consider
  continuing in the obvious way. However after 1. \ldots d5, black has
  refused or not noticed the draw. Fortunately, white is still in a
  good position to play the game normally (this is the main line of
  the R\'eti opening, followed by 2. c4). White can try to be more
  obvious with 2. Ng1, but if black is choosing to just play normally,
  white takes a distinct handicap by doing so.

  The biggest risk for white is that black does not play 2. \ldots Ng8
  but rather a normal move like 2. \ldots g6 (``Fool's Draw Betrayed'').
  This can happen if black is not metagaming at all (for Nf6 is a
  normal response to the normal Nf3), or if black is an exceptionally
  shrewd metagamer (tricking white into wasting two tempos with Ng1
  by pretending to be cooperating).

  Of course, this all relies on the assumption that if
  chess-to-the-death ends in a draw, the players are spared or allowed
  to repeat indefinitely. If both players are actually executed, then
  this line is truly a Fool's Draw!
}

But what if you are forced to {\it be one of the chess pieces} to the
death? That is, your little soul inhabits one of the 32 pieces or
pawns and your soul is vanquished if that piece is eliminated. Now it
doesn't matter whether you're good or bad at chess, because you don't
get to pick what happens in the game. What matters is that your piece
lives to the end of the game, when all surviving pieces are set free.
Which piece should you want to be?

In formal chess, the king can never be captured: The game ends when
the king is attacked but cannot move, and it is illegal to make a move
that leaves the king attacked. The king's death is implied, of course,
but it is seen as more poetic to end the game prior to this point.

For the sake of this question, we'll consider the the white king to
``die'' if white loses (i.e., is checkmated), and likewise for black.
Otherwise, of course, the best chances of survival would trivially be
with the two kings, since they are never formally captured. Loss
includes resignation, since most high-level games actually end once
the defeated player agrees that loss is inevitable. We can think of
this common case like king seppuku. Many games also end in time
forfeit, which is like the king's poor diet and lifestyle choices
leading to a death by natural causes.

Neither side is believed to have a decisive advantage, and many games
end in a draw, with both kings surviving. So the survival chances of a
king are likely greater than 50\%; pretty decent odds. Is it possible
that any other piece has even better chances? Let's find out---our
lives may depend on it!


\footnotetext{
  But is it?
  
  First of all, although either player is allowed to {\it claim} a
  draw after three repetitions of the same position, it is not
  automatic. However, FIDE rules do declare that the game simply ends
  in a draw upon {\it five} repetitions. Of course it is easy to
  extend the Fool's Draw to accommodate this.
  
  Second: The lichess implementation (although known to be
  buggy\cite{kingme}) does not permit a threefold repetition claim in
  this situation, which got me thinking that maybe there is some
  subtlety here. Is the starting position special somehow, not
  counting as having occurred? The relevant statute, from the FIDE
  Laws of Chess\cite{fiderules}:
  \begin{quote}
    {\bf 9.2}.\quad The game is drawn, upon a correct claim by a
    player having the move, when the same position for at least the
    third time (not necessarily by a repetition of moves):
    \begin{enumerate}[label=\alph*.]
    \item is about to appear, if he first writes his move, which
      cannot be changed, on his scoresheet and declares to the arbiter
      his intention to make this move, or
    \item has just appeared, and the player claiming the draw has the
      move.
  \end{enumerate}

  Positions are considered the same if and only if the same player has
  the move, pieces of the same kind and colour occupy the same squares
  and the possible moves of all the pieces of both players are the
  same. Thus positions are not the same if:
  \begin{enumerate}[label=\arabic*.]
  \item at the start of the sequence a pawn could have been captured
    en passant.
  \item a king or rook had castling rights, but forfeited these after
    moving. The castling rights are lost only after the king or rook
    is moved.
  \end{enumerate}
  \end{quote}
  
  So the question is, has the starting position ``appeared'' before
  white's first move? The rules are not totally clear on this point.
  Note that ``positions are considered the same'' only when the
  same player ``has the move.'' FIDE defines ``have the move'' as
  \begin{enumerate}[label=\arabic*.3.] % hack to get 1.3.
    \item A player is said to 'have the move' when his opponent's
    move has been 'made'.
  \end{enumerate}
  A strong case can therefore be made that white does not 'have the
  move' in the formal sense at the beginning of the game, since black
  has not made a move!
  
  Nonetheless, it does seem clear that white can claim a draw by
  9.2.a, by committing the move 5. Nf3 and declaring to the arbiter
  that the position is now {\it about to appear} for the third time.
  This seems unambiguously legal. }

\section{Hypotheses}

Like all good scientific research, I clearly laid out my hypothesis
and wrote down the motivation before performing the study. This helps
prevent presentation bias where the results appear more satisfactory
because they are framed as a natural conclusion from the idea that
motivated the research in the first place (when in fact, of course, if
you write the motivation after witnessing the results, backflow is
inevitable). It is also much more exciting. I literally don't know the
answer as I'm writing this, nor whether it is interesting in any way!

% XXX: Move this stuff to the appendix.

Here are my guesses.

\begin{itemize}
\item Black and white are probably not significantly different. That
  is, the a1 and a8 rooks probably have about the same survival
  chances (it's known that white has a slight statistical advantage
  but it is probably only around 1\%). So these guesses will be
  written about white's pieces.
\item The d2 and e2 pawns are very active in common openings, and
  are frequently captured as part of those openings. I think they
  are the least likely to survive overall.
\item Bishops and knights are often involved in the opening and
  midgame, and often exchanged nonchalantly. I think they all have
  relatively low survival chances.
\item Although the queen is very valuable, a queen exchange is often
  forced for games that enter the endgame.
\item Rooks tend to be late-game pieces, because they are difficult
  to get out of their corners (and at most one can be activated
  by the fastest method, castling) and are relatively valuable.
\item This leaves the non-central pawns. These are the hardest to
  predict, and they are hard to think about (at least for me) because
  when e.g. the a2 pawn recaptures the b3 pawn that it supported, I
  just think of this as the b3 pawn. Of these pawns, b2 and g2 are
  somewhat weak because they are undefended once the bishop is
  developed (cf. the famous ``poison pawn'' at b2). On the other hand,
  in the fianchetto configuration, this pawn is very strong and often
  survives the entire game without leaving the third rank. Since pawn
  chains usually progress towards the middle of the board, the a2 pawn
  is more likely to be supporting than supported. This both leaves it
  weak to capture, but prone to recapturing. Outside pawns block one's
  own rook, although for this same reason they often clear the file by
  capturing (and so survive). They are also commonly used to push into
  a well-defended king's territory (e.g. in the fianchetto); kingside
  castling is more common, so this means that the h pawns are often
  lost to this fate.
\end{itemize}

This leaves my final ranking, from most surviving to most dead:

\begin{enumerate}
\item  f2
\item  c2
\item  g2
\item  a2
\item  h2
\item  b2,
\item  h1 rook,
\item  a1 rook,
\item  king,
\item  queen,
\item  f1 bishop,
\item  c1 bishop,
\item  g1 knight,
\item  b1 knight,
\item  e2,
\item  d2,
\end{enumerate}

As already copped to, while the author is an aficionado and also knows
how to spell the difficult word aficionado without spell-check, he is
not good at chess. A few drinking buddies with varying chessperience were
also consulted for their wagers.

\subsection{Ben}\footnote{Ben does not prefer to use the shift key.}

edge pawns almost never played til endgame let alone traded off

ph
pa

not quite sure where these should go (pb more likely to see play in
queenside minority attacks in k-side castle games?)

pf
pg
pb

rook play more likely to be active on q side than on k side (also the
classic Nxc7 fork in low rank play), but overall more likely to stay
tucked away compared to q

Rh
Ra

i think IQP positions are more likely than not saccing e in e4 openings
but on the other hand d is often traded off in e4 openings while vice
versa is not as true

pe
pd

q probably involved in many checkmates (low ranked play) or resignations
before traded off (high ranked play)

Q

just randomly guessing k dies in about 1/3 of games, times 1/2 for 2
sides

K

this pawn is a super goner (sicilian, QGA, ...)

pc

most doomed seem to be the minor pieces as i'd guess at least half of
them get traded off on c/f/3/6 or e/d/4/5 in near every game so

Bc
Bf
Nb
Ng

\subsection{Jim}

Most-to-least-survival hero tier list for chess (patch 1.0):

1.: King
 -- If I estimate that about 2/3rds of all regular pieces are captured in an average game, and the probability of any non-king piece being captured is uniform, then the king is clearly the most likely to survive. (I'm going to break symmetry here and rank black king less likely to survive than white king.)

2. Both Rooks
 -- Kept in reserve for castling purposes.

3. The A, B, G, and H pawns
 -- maybe people will forget to move them because they are far from the center.

4. Both Knights
 -- They are slippery, but they often get deep into enemy territory quickly.

5. The C,D,E,F pawns
 -- Moved forward to release various more important pieces $\Rightarrow$ more likely to die.

6. Both Bishops
 -- https://youtu.be/gDnE-5lD7w8

7. The Queen
 -- A high-value target, seems unlikely to survive.

This list was assembled for chess.report by ix, who only barely
understands the rules of chess and rarely plays. This is probably why
the justifications get increasingly nonsensical.

\subsection{David}

Here's my ranking, along with the probabilities of survival that I guess for each piece:

white g-pawn: 0.72
black g-pawn: 0.72
white b-pawn: 0.69
black a-pawn: 0.66
black b-pawn: 0.65
white king: 0.65
white a-pawn: 0.64
black king: 0.55
black f-pawn: 0.54
white f-pawn: 0.54
black h-pawn: 0.53
white h-pawn: 0.53
white kingside rook: 0.52
black kingside rook: 0.52
white queenside rook: 0.51
black queenside rook: 0.51
white c-pawn: 0.49
black c-pawn: 0.44
white light-squared bishop: 0.33
black light-squared bishop: 0.33
white dark-squared bishop: 0.32
black dark-squared bishop: 0.32
white queen: 0.31
black queen: 0.30
white kingside knight: 0.29
black kingside knight: 0.29
white queenside knight: 0.28
black queenside knight: 0.28
white e-pawn: 0.19
black d-pawn: 0.18
white d-pawn: 0.17
black e-pawn: 0.16

\subsection{William}

Okay, this is pretty much off the cuff, but let's say, just for white,
from most to least likely to survive:

Queenside Rook
B Pawn
A Pawn
Kingside Bishop
C Pawn
Queen
King
H Pawn
Queenside Bishop
F Pawn
G Pawn
Queenside Knight
Kingside Rook
Kingside Knight
D Pawn
E Pawn

Some motivated reasoning:

I figure the king has got to be somewhere near the middle of the pack
since he dies in half of games featuring a winner---but with slightly
higher-than-even odds of surviving, since some games end in a draw.
I’m probably mixing up means and medians here somehow..

...
King
...

I’m gonna assume castling happens more often on the King’s side, so
let’s give Kingside Rook and F, G, and H Pawns a better shot than
their fellows on the left. But maybe it should actually be worse,
since if they die, it's because they failed to protect the king. Plus,
having heard the tip about C Pawn loud and clear, I’m gonna assume
that bad boy most often becomes a new Queen, which means he gets more
survival points than the real Queen herself.

...
C Pawn
Queen
A Pawn
B Pawn
Queenside Rook
King
F Pawn
G Pawn
H Pawn
Kingside Rook
...

D and E Pawn are nothing but pawns, and they mostly sacrifice
themselves to the cause.

...
D Pawn
E Pawn

Randomizing within these constraints gives us our starting point. Then
the wildcard Bishops and Knights get randomly distributed through what
remains to come up with this final answer shown above.

\section{Methodology}

To compute the chances for survival, I legally acquired 506,000,416
chess games from lichess.org. This is all of the standard variant,
rated games from Jan 2013 to November 2018, in any time format. Only
games that are completed and valid are included (about 200,000 games
did not meet this criteria). The total data size is 875 gigabytes, so
processing these took some care for efficiency and parallelism.

Other than that, I simply parsed and simulated each game. For each
of the 32 pieces in the starting position, I tracked its current
location, and whether it is alive; multiple dead pieces could
occupy the same square. At the end of the game, one of the kings is
killed if his side has lost.

For a piece, there is a single factual survival rate in these games,
given just by $\frac{\mathrm{num~survived}}{\mathrm{num~games}}$. What
we're really interested in, however, is estimating the underlying true
survival probability for each piece. In order to do this with
reasonable efficency, we divide the games into 32 separate buckets,
and count statistics separately for each. From these samples we can
then estimate variance, for example. The games are bucketized by a
deterministic hash of the White player's username. This way, if there
exist some players who are highly unusual (perhaps automated
accounts), their games are grouped together and pessimistically
represented in the variance estimate. This also helps account for
different opening preferences; the chosen opening certainly affects
the survival chances.

\begin{figure}[htb]
  \begin{center}
    % To generate this pdf, I ran maketable to make
    % piece-survival.svg, then printed to PDF using Chrome, then
    % cropped it in Illustrator. Illustrator can open the SVG but
    % doesn't have the unicode chess pieces in any font I could find.
    \includegraphics[width=0.95 \linewidth]{piece-survival-export}
  \end{center}\vspace{-0.1in}
  \caption{Survival probabilities for each of the 32 pieces in standard
    chess, in 500 million games. The number is the mean survival rate
    across all samples. The vertical position of the dot is this mean
    rate, with a line drawing the span between the smallest and largest
    sample bucket (this is usually a very tiny range).
    Horizontal position is purely presentational, to avoid overlap.}
  \label{fig:piece-survival}
\end{figure}

The basic survival chances appear in Figure~\ref{fig:piece-survival}.
Indeed, many pieces are more likely to survive than the kings. Even as
black, the extremal pawns (\pawn a and \pawn h) have over a 70\%
survival rate. Across the board, the survival chances for a white
piece and its black twin are similar, usually with a small edge to
white. Notable exceptions are the \knight g (the overall most doomed
piece), and both white bishops, which die more than their
Schwarzdoppelgangers. The \pawn e is vastly more dead than \Pawn e.
Note somewhat satisfyingly that the \pawn c has the highest variance;
this was the most controversial among the drinking buddies
(Section~\ref{sec:XXXbuddies}). Note that \pawn c is the sacrificed
pawn in the popular Queen's Gambit (1. d4 d5 2. c4), where accepting
and declining the pawn are both popular and sound responses. This may
be a good example of a piece that has substantially different survival
rates in different opening preferences. Since the variance is
otherwise extremely low, I only report means for the remainder of the
paper.

Despite my impression that many games end in a draw, ties are actually
rare in the lichess database. In January 2018, only 3.8\% of games
were drawn;
% (8922275 white, 8329221 black, 682042 draw)
as a result, the survival chances for the kings are both close to
50\%. Although the database contains games in many time formats and
with all varieties of human skill (including over a thousand games by
Magnus Carlsen, the world champion and highest rated player of all
time\footnote{Although to be fair, his username ``DrDrunkenstein''
  suggests he may not play at full strength.}), blitz ($\sim 5$ minutes
per side) and bullet ($\sim 1$ minute) games are predominant. In
Section~\ref{sec:comparative} I show some slices of the data for
comparison. % XXX could summarize whether this is a real effect or
% not, once I have the slices!

\section{Safest spaces}

The fate of each piece is to either survive or die, and it does so on
one of the 64 squares. With the same replay of the 500 million games
I also kept statistics on the fates of each piece. If being a chess
piece to the death, and possessing some influence over where your piece
moves, it may be helpful to know where to go. Even without influence,
such knowledge could help calibrate your anxiety.

Other than the bishops---which have no legal way to reach half of the
squares---every piece ends on every square in at least a thousand
games. So we have enough samples to have reasonable confidence in our
statistics, even for the most unlikely odysseys. The least mobile
pieces are the pawns, who can technically reach any square by
promoting, but are usually confined to cones emanating from their
start squares. The overall rarest fate is for \pawn f2 to die on the
a7 square, which only happened 1,244 times (however, it survived on
this square 31,438 times). This square is actually reachable without
promoting, but it would need to capture 5 times in order to get
there, which seems quite unlikely! There may even be a hidden
achievement for reaching this square this way!
% TODO: Perhaps could find a game where this happened?!
Aside from pawns, the weirdest fate is for \knight g to die on h1,
which happened 47,307 times. Corners are of course fairly garbage for
the knight, although it is twice as likely to survive on this square.

% There may even be hidden achievements for reaching these squares!

That said, there are characteristic patterns for each of the pieces,
which make sense given their starting positions and patterns of
movement. You could probably guess the piece just by looking at one of
the heat maps below, although---spoiler alert---the piece is just
listed right there and they are in order. Two independent things are
communicated in these graphics: The chance that the piece ends the
game on some square (alive or dead), and its survival chances there.
In each map, a darker background color indicates that the piece ends
on that square more often. The shade is based on the rank (64/64 black
is most common, 63/64 black is next most common, etc.) rather than the
absolute probability, since otherwise the graphic looks boring. The
number on the square is the percentage survival of the piece when it
is last seen (either being captured or surviving to the end of the
game) on that square. The four squares with the highest survival rates
are underlined for your convenience.

\newcommand{\chessmap}[2]{\begin{center} \includegraphics[width=0.65 \linewidth]{#1} {#2} \end{center}\vspace{-0.1in}}

\chessmap{piece0}{\Rook a8}
\chessmap{piece1}{\Knight b8}
\chessmap{piece2}{\Bishop c8}
\chessmap{piece3}{\Queen d8}
\chessmap{piece4}{\King e8}
\chessmap{piece5}{\Bishop f8}
\chessmap{piece6}{\Knight g8}
\chessmap{piece7}{\Rook h8}

\chessmap{piece8}{\Pawn a7}
\chessmap{piece9}{\Pawn b7}
\chessmap{piece10}{\Pawn c7}
\chessmap{piece11}{\Pawn d7}
\chessmap{piece12}{\Pawn e7}
\chessmap{piece13}{\Pawn f7}
\chessmap{piece14}{\Pawn g7}
\chessmap{piece15}{\Pawn h7}

\chessmap{piece16}{\pawn a2}
\chessmap{piece17}{\pawn b2}
\chessmap{piece18}{\pawn c2}
\chessmap{piece19}{\pawn d2}
\chessmap{piece20}{\pawn e2}
\chessmap{piece21}{\pawn f2}
\chessmap{piece22}{\pawn g2}
\chessmap{piece23}{\pawn h2}

\chessmap{piece24}{\rook a1}
\chessmap{piece25}{\knight b1}
\chessmap{piece26}{\bishop c1}
\chessmap{piece27}{\queen d1}
\chessmap{piece28}{\king e1}
\chessmap{piece29}{\bishop f1}
\chessmap{piece30}{\knight g1}
\chessmap{piece31}{\rook h1}

\section{Guesses and slices}
%  
%  \begin{tabular}{|l|l|}
%  \hline
%    tom & actual \\
%   \hline
%   \Pawn f    &  \\
%   \Pawn c    &  \\
%   \Pawn g    &  \\
%   \Pawn a    &  \\
%   \Pawn h    &  \\
%   \Pawn b    &  \\
%   \Rook h    &  \\
%   \Rook a    &  \\
%   \King      &  \\
%   \Queen     &  \\
%   \Bishop f  &  \\
%   \Bishop c  &  \\
%   \Knight g  &  \\
%   \Knight b  &  \\
%   \Pawn e    &  \\
%   \Pawn d    &  \\
%  \hline
%  \end{tabular}
%  

% Battle chess
% Feminist game 

\nocite{chesstego}

\bibliography{chess}{}
\bibliographystyle{plain}

\end{document}
