\documentclass[twocolumn]{article}
\usepackage[top=1.1in, left=0.85in, right=0.85in]{geometry}

\usepackage{supertabular}
\usepackage{amsthm}
\usepackage{relsize}
\usepackage{amsmath}
\usepackage{amssymb}
% \usepackage{code}
\usepackage{graphicx}
\usepackage{fancyvrb}
\usepackage{url}
\usepackage{textcomp}

\pagestyle{empty}

\newcommand\comment[1]{}

\newcommand\st{$^{\mathrm{st}}$}
\newcommand\nd{$^{\mathrm{nd}}$}
\newcommand\rd{$^{\mathrm{rd}}$}
\renewcommand\th{$^{\mathrm{th}}$}
\newcommand\tm{$^{\mbox{\tiny \textsc{tm}}}$}

% nice fractions
\newcommand\sfrac[2]{{}\,$^{#1}$\!/{}\!$_{#2}$}

\newcommand\citef[1]{\addtocounter{footnote}{1}\footnotetext{\cite{#1}}\ensuremath{^{\mbox{\footnotesize [\thefootnote]}}}}

\usepackage{ulem}
% go back to italics for emphasis, though
\normalem

\begin{document} 

\title{The glEnd() of Zelda}
\author{Dr.~Tom~Murphy~VII~Ph.D.\thanks{
    Copyright \copyright\ 2016 the Regents of the Wikiplia Foundation.
    Appears in SIGBOVIK 2016 with the danger of going alone of the
    Association for Computational Heresy; {\em IEEEEEE!} press,
    Verlag--Verlag volume no.~0x2016. 255 Rupees} }

\renewcommand\>{$>$}
\newcommand\<{$<$}

\date{1 April 2016}

\maketitle

\section*{Abstract}
3D ZELDA

\vspace{1em}
{\noindent \small {\bf Keywords}:
  small keys, boss keys, dungeon keys
}

\section{Introduction}

{\bf 1986. Hyrule.}\quad The Legend of {\it frickin'}~Zelda for the
Nintendo {\it freakin'}~Entertainment System. Need I say more? A {\it
  god damn} \uline{gold cartridge}. Fortunes made just from melting
the gold cartridge down to make gold teeth grilles, after carefully
extracting and preserving the even more precious ROM inside. A die-cut hole in the
box so that you could get a peek of the cartridge and presage that you
were getting some solid gold. A die cut little window that you could
palpate through the wrapping paper on Christmas~Eve, presaging some
epic thumb blisters in store for the coming weeks. Koji {\it freckin'}
Kondo. Koji Kondo whipping up a nice 8-bit arrangement of Bol\'ero as
the theme music until realizing at the last minute that this music was
copyrighted\footnote{Perhaps ironic, since Ravel's {Bol\'ero} itself
  was composed as a result of Ravel getting
  cop-blocked.\cite{wikipedia2016bolero} And surely some Zelda
  knock-off since then has included a Muzak ersatz of the Zelda theme!
  What is the longest documented sequence of compositions due to
  Copyright restrictions?} and so instead composing its epic theme in
{\it one day}??

\begin{figure}[ht]
\begin{center}
\includegraphics[width=\linewidth]{link3d}
\end{center}\vspace{-0.1in}
\caption{Technical diagram of mathematical equations.} \label{fig:link3d}
% \smallskip
% caption can go here...
% } 
\end{figure}

A gold cartridge that contained ROMs and a little swallowable battery
to keep the onboard SRAM powered up so that it could retain your epic
save game. A battery designed to last 70~years. Nothing could cause
you to lose your save game, even once you were half way through the
Second Quest. Unless your little brother starts a completely new game
and saves over your slot, earning him one of the most righteously
deserved clobberings this side of Inigo Montoya. Saves right on top of
your slot with a player called just \verb+A +. Right over your slot,
erasing it, and hasn't even picked up the {\it sword} yet. All you
have to do is step on the black square in the first room and the guy
gives you the sword and then dies. Jeez pooleaze. Saves right over
yours with no sword even though there are TWO OTHER UNUSED SLOTS and
you even wisely put your game in the third slot {\it exactly to avoid
  this kind of calamity}.

\medskip
Need I say more? Shall I say it? Apparently so. KIDS THESE DAYS. Kids
these days and their three dimensions. You can certainly make a
respectable case that the Zeldest of the series was {\it Legend of
  Zelda III: A Link To The Past}, or even {\it Legend of Game Boy
  Zelda: I forgot What It's Called}. But there seems to be an
established consensus that {\it Legend of Zelda LXIV: The Ocarina of
  Time} is the true masterpiece; that while the NES Original indeed
invented the genre, the technical limitations of the system
handicapped it. You're talking to me about technical {\it frockin'}
limitations? Like somehow appreciating the subtle wonder of the
universe and its mechanics requires me to look at 3D tree people with
tetrahedric clubs for hands. And they say this of the 64\th\ game in
the series? Please jouize.

A fine game, don't get me wrong. But was its theme song composed in
one day? Did it come in a gold cartridge?

And the thing is: The original Zelda was built with enough forethought
that it {\em can} be rendered in first-person 3D. So here I will make
the game 3D so that it can be enjoyed by these kids and they can get
off my lawn, tapdancing around playing their titular Ocarinas. I will
make it possible to play the game in glorious HD, or even 4K
resolution, and then some (Figure~\ref{fig:link3d}). And while
we're at it, why not try to do it in a way that can play loads of NES
games in three {\it fr\"ockin'} dimensions? What's the worst that
could happen?


\section{Technical limitations}

As presaged, the NES does have certain technical limitations, which I
prefer to think of as constraints. Some basic details are important
for understanding how this works.

The NES has a modest CPU that executes game logic. It's a little 8-bit
baby puppy with access to 2 kilobytes of RAM. If you think about it,
the NES's 256x240 pixel screen is already 61,440 pixels, which if it
used the entire 2k of RAM to represent it, we'd have only
$0.2\overline{6}$ bits of RAM per pixel, which makes no sense. At XXX
MhZ, and a generous 1 instruction per pixel, we'd only be able to render
XXX frames per second! This doesn't make sense, and this is because
the CPU does not render the graphics on a NES.

The NES also contains a custom Picture Processing Unit, or PPU, the
8-bit baby puppy's baby ppupy brother. This thing outputs 60 frames of
NTSC or PAL video per second. In fact, since the PPU also drives a CPU
interrupt for each scanline and vblank (XXX true?), and most games use
this for timing,%
%
\footnote{Fun fact: NTSC, the video standard used in the Americas, has
  % XXX how u make ``approx'' symbol
  a \~60Hz frame rate. But PAL, used in Europe and Asia, has a 50Hz
  refresh rate. This means that many games run 16\% slower in Europe
  than the US, so if you grew up playing there, you were playing on
  {\em Easy} mode. {\tt\ ;-)}}
%
in many ways the CPU is subservient to the PPU, kind of like the PPU
is the {\it true} pure heart of the NES. In the context of this paper, you
could say that the CPU is the Ocarina of Time and the PPU is the
Legend of Zelda. It's just an analogy. The PPU is very complicated and
has tetrahedra for hands, like clubs. The CPU communicates with
the PPU not by drawing pixels, but by giving a high-level description
of the screen (by writing it into PPU memory). The chief insight of
this work is that the PPU representation can be interpreted and
rendered in 3D instead of 2D. This allows us to use the exact original
game logic and only change the way it's viewed. Sometimes this even
allows us to see aspects of the game that are actually present but
were not visible on NES hardware. Often it makes the game more
difficult by making things invisible (enemies behind the player or
behind walls) and inducing motion sickness.

\subsubsection{Background}
The PPU has two major drawing facilities: Background and Sprites. Both
are made of ``tiles'', which are 8x8 graphics that are
usually\footnote{Some games trick the NES hardware by remapping tile
memory (e.g. in response to a memory read), even during a scanline. I
do not allow for such chicanery in this work.} stored in ROM. There
are two tile sets, each with 256 tiles. Figure~\ref{fig:ppu} shows the
tile sets for the starting location in the Zelda overworld.

\begin{figure}[ht]
\begin{center}
\includegraphics[width=\linewidth]{ppu1} \\
\vspace{1em}
\includegraphics[width=\linewidth]{ppu2}
\end{center}\vspace{-0.1in}
\caption{Pattern tables for the Zelda overworld. In Zelda, the top tiles
are used for sprites, and the bottom are used for the background graphics.
All tiles are 8x8 and use 4 color values (color 0 is transparent). These
color values index into 4 sprite palettes and 4 background palettes, with
some limitations.
} \label{fig:ppu}
\end{figure}

The NES background is described by the ``nametable'', which is an
array of 32x30 bytes giving the tile numbers to fill the screen with.
It's two rows shy of 256 probably for some reason having to do with
the aspect ratio of TVs or number of actual NTSC scanlines, but also
to make room for 64 bytes of attribute data. There are just two bits
per four tiles. These two bits select which of the four palettes to
use for each 2x2 group of tiles on the screen. It's an interesting hobby to
look for the tricks that artists use to work around the constraint on
the number of background colors. % XXX figure showing this off?

Because of this array of tiles, NES games have to be built of
high-level, block-like structures. In Zelda, this structure is
particularly (but not unusually) regular. Since each 2x2 tile group
(16x16 pixels) shares the same palette, the screens are built around
16x16 pixel grid-aligned blocks.

\subsubsection{Sprites} \label{sec:sprites}

Sprites, named in honor of the forest pixies whose homes were razed in
order to make space inside the PPU for them, are more complicated.

There are 64 sprites, which are all turned on or off together. Each
sprite is described by four bytes: Its $x$ and $y$ coordinates (these
can be placed at arbitrary integer pixel locations), the index of
the tile to draw, and an attribute byte:

\begin{tabular}{rrl}
    & bit \# & bit deets \\
\hline \\
MSB & 7      & Vertical flip \\
    & 6      & Horizontal flip \\
    & 5      & something weird \\
    & 4      & I don't know \\
    & 3      & unused? who can say \\
    & 2      & bit 2; could be anything \\
LSB & 1,0    & Palette index \\
\end{tabular}

Additionally, there are some global sprite settings. The only one that
is important for this presentation is the ``tall sprites'' flag. If
it's set, then sprites are 8x16 (drawn of two consecutive tiles)
instead of 8x8. Sprites are drawn in a particular order. Barring some
bizarre mischief, all of the sprites are drawn; you can't turn them
off individually. Instead, you position the ones you don't want off
the screen or make them be completely transparent. In order to make
things more challenging, if there are exactly 8 sprites on a
particular scan line, some really weird stuff happens. % XXX explain

Sprite Zero has some special properties. I don't want to tell you
about the properties, I just wanted to make a soda joke.

\section{Three to D---$u$ and $v$}\footnote{Spoiler alert because I think
this joke may be too abstruse. ``Free to Be You and Me'' was a
semi-famous kids record (here meaning actual vinyl) by 
% XXX
Milo Thomas. D is ``dimensions.'' $u$ and $v$ are common names of
the axes for texture coordinates in computer graphics.}

The game runs in an emulator, a version of FCEUX\cite{fceux} that I
have hacked up almost beyond recognition. In normal operation, it

\begin{itemize}
\item Emulates a complete frame of the NES. This includes the CPU and
  PPU operation for all 240 scanlines. The execution of a frame depends
  on the current controller input (which doesn't change as it runs);
\item Extracts scene geometry and textures. This consists of two
  pieces: Boxes and Sprites;
\item Extracts the player location and orientation;
\item Moves texture data to the GPU;
\item Transforms the scene geometry into a textured model, sets the camera
  based on the player location, and renders.
\end{itemize}

I'll describe the interesting parts of these.

\paragraph{Emulation.} 
Well, this is not that interesting. I give the emulator its controller
input and it does its thing, modifying the RAM and PPU and CPU state.
I have been working on improvements to emulator technology but these
are independent of this paper and better described elsewhere.

\paragraph{Extracting boxes.} 
Yeah! Okay! Boxes are the three-dimensional version of squares, and
are what give the game its 3Dness. I read the nametable in chunks of
2x2 tiles, which is the size of one overworld tree (and the same size
as Link). Recall that Zelda uses this grid size, probably because
color attributes can only be set at this scale. These become $2x2x2$
cubes. The $x$ and $y$ coordinates are obvious (pixel positions
divided by 16, since each 2x2 tile block is 16x16 pixels), but what
should $z$ be? Zelda and other games tend to use tiles quite
consistently to represent walls, obstacles, floor, empty space, and so
on. This makes sense because a tree graphic, for example, suggests
something to the player. So we simply build a file,
\verb+zelda.tiles+, which maps tiles (by index) to one of \verb+wall+,
\verb+floor+, \verb+rut+, or \verb+unknown+. Floor blocks are placed
at $z=0$, which will be under the player's feet (we typically only see
the top face). Walls are placed at $z=2$, so that their bottoms are
coplanar with the floor. Rut blocks are placed at $z=-0.5$; this is
used for blocks that are part of the ``floor'' but that can't be
walked on, like water. Because each box is actually made of 4 tiles,
we just use the maximum $z$ value for the constituent tiles (they should
normally be the same). It would make sense to develop heuristics for
when games use tile-level resolution, even to generate primitives
other than cubes.

This approach is a bit unsatisfying because it requires us to have
specific knowledge of the game as well as do some manual labor.
However, there really aren't that many tiles, and it's kind of fun.
(In fact, you really only need to pick out a handful of floor tiles
before the game is totally playable.) There is another problem, which
is that many games switch tile sets, which may change whether tile
\verb+0x84+ (``desert dots'' in the overworld) is floor or wall. What
happens if you go into a dungeon in Zelda, for example? I didn't even
try that yet but I bet it's nuts! This can easily be fixed by
identifying distinct tile sets (for example by using the source ROM
address, or a hash of their content) and mapping them all.

However, there's some hope that this can be done in a much more
automatic way. I describe an idea in Section~\ref{sec:cast}.

\paragraph{Box textures.}
The game is pretty understandable with just TRON-cubes, but it is
better to texture them. We put the same texture on each face of a box;
that texture is just the 16x16 pixel graphic consisting of the 4 tiles
that made it, in the appropriate palette. It does {\it not} work to
generate a bunch of tile-sized boxes (with the individual tile texture
on all sides); these will only look right on the top face. Indeed, it
is possible to get weird texture mismatches between these 2x2x2 boxes
for basically the same reason. It usually looks pretty good, though.

Conceivably, every box on the screen could have a different texture,
and conceivably these all change every frame. It is inefficient to
move 960 distinct textures to video memory every frame, so instead we
actually just move the rendered background (256x240) itself and record
with each box its texture coordinates within. This is {\it much}
faster, though it can create artifacts at the edges of boxes as an
adjacent pixel peeks into the texture. I checked out Ocarina of Time
and it is {\it loaded} with texture seam issues, so I figure this is
how they like it.

\paragraph{Extracting sprites.}
Sprites are the most difficult. Aside from just being more fiddly to
implement, it is tricky to recover an important high-level structure.
Although we know the location of each individual sprite and the two tiles
(Zelda uses ``tall' 8x16 sprites; see Section~\ref{sec:sprites}) that
make it up, most game objects (like enemies) are in fact drawn with
{\it two} sprites. These sprites could be in any of the 64 sprite
slots (it's not at all obvious how Zelda allocates these, or that it
would be similar for other games). Adding to the complexity, sprites
like Link's sword or the pulsing balls of plasma that frogmen shoot
at you from the water {\it just for being an elf} are just a single
sprite, and bosses can be like XXX sprites (XXX add figure). If we
promote sprites to 3D without knowing which ones are supposed to be
part of the same object, then the enemies get cut up along their
sprite seems and it looks super wrong.

To fix this, we perform {\it sprite fusion}. We say that two sprites
are part of the same object if they share a complete edge
(transitively). Since this allows the construction of very tall ($64
\times 16$) or wide ($64 \times 8$) sprites, we have to make sure that
our textures are large enough to contain large sprites. Rather than
try to pack them into a texture at runtime (computationally expensive
since they can also be weird shapes), we just pre-allocate one texture
of the maximum size for each sprite slot (16MB video ram; nothing).
Sprite fusion, computation of the sprite size, drawing the sprites,
extracting the geometry, and moving the textures to the video card are
all intertwined. The simplified algorithm is

\begin{itemize}
\item Create 64 ``pre-sprites'', which are baby sprites. These know
  their coordinates and have the index of a ``parent'' pre-sprite
  (initialized to $-1$).
\item While filling these in, consult a hash-map to see if there are
  already pre-sprites exactly aligned in each of the cardinal
  directions. If so, use a union-find--like algorithm (XXX cite)
  to efficiently update its parent index; each fused sprite has one
  canonical representative whose parent index is $-1$. Add the
  pre-sprite to the hash-map.
\item Another pass pushes the min/max bounds to the canonical representative,
  so that we know how big the fused sprite is.
\item Another pass draws all sprites into the texture data for their
  canonical representative, at the appropriate location. This pass must
  happen in the same order as on the NES hardware, so that overlay effects
  are correct.
\item Finally, create proper sprites to return for each of the fused sprites.
  Copy the used region of the texture data for these to the video card.
\end{itemize}

% XXX diagram of this algorithm would be nice

Two types of sprite rendering are supported: in-plane and billboard.
In-plane just draws the sprites on the floor or some other $xy$ plane
of your choice. (This does not actually need sprite fusion, but it is
not harmful.) Billboard sprites are drawn so that they face the
player, like in Wolfenstein~3D. These are positioned at their $xy$
centroid and positioned in $z$ so that the bottom of the sprite is
touching the floor.

\paragraph{Extracting location.}

Most NES games use very simple layouts for their game facts (i.e.,
global variables at fixed memory locations), since the processor is
simple and memory light.\cite{murphy2013first} In Zelda, Link's
$x$ coordinate is at \verb+0x70+ and $y$ at \verb+0x84+. Since this
is his top-left corner, we add 8 to each to get to the middle of the
16x16 pixel boy.

We also have to face in the same direction as link, to make the
perspective first-person. Link's orientation is at memory location
\verb+0x98+.

Link moves smoothly (with pixel resolution) around the screen, but
turns abruptly. This is sickening to watch. I apply some smoothing
to Link's angle, ...

(This also involves manual steps..)

\paragraph{Rendering.}

......

Note that some of the tiles on screen are used to draw a menu at the
top, which is not part of the gameplay area. This still gets interpreted
as 

Problem with link always hovering behind you

\section{A Link to the cast} \label{sec:cast}

\nocite{murphy2013first}

\bibliographystyle{IEEEtran}
% \bibliographycomment{} % nothing
\bibliography{paper}

\end{document}
