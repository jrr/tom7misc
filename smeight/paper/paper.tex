\documentclass[twocolumn]{article}
\usepackage[top=1.1in, left=0.85in, right=0.85in]{geometry}

\usepackage{supertabular}
\usepackage{amsthm}
\usepackage{relsize}
\usepackage{amsmath}
\usepackage{amssymb}
% \usepackage{code}
\usepackage{graphicx}
\usepackage{fancyvrb}
\usepackage{url}
\usepackage{textcomp}

\pagestyle{empty}

\newcommand\comment[1]{}

\newcommand\st{$^{\mathrm{st}}$}
\newcommand\nd{$^{\mathrm{nd}}$}
\newcommand\rd{$^{\mathrm{rd}}$}
\renewcommand\th{$^{\mathrm{th}}$}
\newcommand\tm{$^{\mbox{\tiny \textsc{tm}}}$}

% nice fractions
\newcommand\sfrac[2]{{}\,$^{#1}$\!/{}\!$_{#2}$}

\newcommand\citef[1]{\addtocounter{footnote}{1}\footnotetext{\cite{#1}}\ensuremath{^{\mbox{\footnotesize [\thefootnote]}}}}

\usepackage{ulem}
% go back to italics for emphasis, though
\normalem

\begin{document} 

\title{The glEnd() of Zelda}
\author{Dr.~Tom~Murphy~VII~Ph.D.\thanks{
    Copyright \copyright\ 2016 the Regents of the Wikiplia Foundation.
    Appears in SIGBOVIK 2016 with the danger of going alone of the
    Association for Computational Heresy; {\em IEEEEEE!} press,
    Verlag--Verlag volume no.~0x2016. 255 Rupees} }

\renewcommand\>{$>$}
\newcommand\<{$<$}

\date{1 April 2016}

\maketitle

\section*{Abstract}
3D ZELDA

\vspace{1em}
{\noindent \small {\bf Keywords}:
  small keys, boss keys, dungeon keys
}

\section{Introduction}

{\bf 1986. Hyrule.}\quad The Legend of {\it frickin'}~Zelda for the Nintendo {\it freakin'}~Entertainment System. Need I say more? A {\it god damn} \uline{gold cartridge}. Fortunes made just from saving the gold cartridge and melting it down to make gold teeth grilles, after carefully extracting the even more precious ROM inside. A die-cut hole in the box so that you could get a peek of the cartridge and presage that you were getting some solid gold. A die cut little window that you could palpate through the wrapping paper on Christmas~Eve, presaging some epic thumb blisters in store for the coming weeks. Koji {\it freckin'} Kondo. Koji Kondo whipping up a nice 8-bit arrangement of Bol\'ero as the theme music until realizing at the last minute that this music was copyrighted\footnote{Perhaps ironic, since Ravel's {Bol\'ero} itself was composed as a result of Ravel getting cop-blocked.\cite{wikipedia2016bolero} And surely some Zelda knock-off since then has included a Muzak ersatz of the Zelda theme! What is the longest documented sequence of compositions due to Copyright restrictions?} and so instead composing its epic theme in {\it one day}??

A gold cartridge that contained ROMs and a little swallowable battery to keep the onboard SRAM powered up so that it could retain your epic save game. A battery designed to last 70~years. Nothing could cause you to lose your save game, even once you were half way through the Second Quest. Unless your little brother starts a completely new game and saves over your slot, earning him one of the most righteously deserved clobberings this side of Inigo Montoya. Saves right on top of your slot with a player called just \verb+A   +. Right over your slot, erasing it, and hasn't even picked up the {\it sword} yet. Saves right over yours even though there are TWO OTHER UNUSED SLOTS and you even wisely put your game in the third slot {\it exactly to avoid this kind of calamity}. 

Need I say more? Shall I say it?


technical limitations...
Like looking at tree people with polyhedra for hands is some kind of ...

\begin{figure*}[p]
\begin{center}
\includegraphics[width=3.5in]{link3d}
\end{center}\vspace{-0.1in}
\caption{Technical diagram of mathematical equations.} \label{figure:link3d}
\smallskip
% caption can go here...
% } 
\end{figure*}

\nocite{murphy2013first}

\bibliographystyle{IEEEtran}
% \bibliographycomment{} % nothing
\bibliography{paper}

\end{document}
